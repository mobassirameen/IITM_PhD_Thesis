%How to use:
%To work on thesis go to the thesis.tex page and compile it. MS and PhD title pages are different. Set the documentclass option to MS or PhD as per the requirement as shown below. For synopsis, compile synopsis.tex.
%To update title of the thesis, name, month and year, goto titlepage_ms(phd) and innertitle_ms(phd) and update on corresponding places. If title of the thesis needs to be divided in multiple lines then follow the pattern as commented on that page. We can not use the \centering option of latex due to left border.
%Line spacing and margins will be adjusted automatically by the template.
%Sometimes image which are imported have some whitespaces around them. Adjust the top and bottom spaces of such images manually ensure compliance with the format.
%Make sure to write the captions of figures and tables as suggested in the template/by the department. There are examples of tables and figures in this template with the suggested spacing. 
%for verification please refer to the latest guidelines as per insti available at https://academic.iitm.ac.in/getpdf.php?id=422 
%for reference the prefered format is last name of the first author and the year. Use \citep for the same. Check iitm.bst file for details of reference formats.

%suggested optimizations
%Title and inner title pages are similar except the copyright sign at the bottom of innertitle page. They are designed separately as titlepage_ms.tex and innertitle_ms.tex (similarly for phd). 
%easier method for modifying title of the thesis, name and month, year.

%\documentclass[PhD]{iitmdiss}
\documentclass[MS]{iitmdiss}
%\documentclass[MTech]{iitmdiss}
%\documentclass[BTech]{iitmdiss}
\usepackage{times}
\usepackage{t1enc}

\usepackage{graphicx}
\usepackage{hyperref} % hyperlinks for references.
\usepackage{amsmath} % easier math formulae, align, subequations \ldots
\usepackage{afterpage}
\newcommand\blankpage{%
    \null
    \thispagestyle{empty}%
    \addtocounter{page}{-1}%
    \newpage}
    
\usepackage{tikz}
\usepackage{booktabs,multirow,tabularx}
\usepackage{anyfontsize}

\usepackage{titlesec}

\usepackage{setspace}
\titleformat{\section}
  {\vspace{1.2em}\normalfont\fontsize{12}{0em}\bfseries}{\thesection}{1em}{\vspace{-0.6cm} \MakeUppercase}
  
\titleformat{\subsection}
  {\vspace{1.2em}\normalfont\fontsize{12}{0em}\bfseries}{\thesubsection}{1em}{\vspace{-0.5cm}}
  
\setlength{\parindent}{0em}
\setlength{\parskip}{1.2em}

\setstretch{2.0}

%---------------------------------
%to avoid hyphenation in text
\tolerance=1
\emergencystretch=\maxdimen
\hyphenpenalty=10000
\hbadness=10000

\sloppy
%to avoid hyphenation in text
%---------------------------------

\begin{document}

%%%%%%%%%%%%%%%%%%%%%%%%%%%%%%%%%%%%%%%%%%%%%%%%%%%%%%%%%%%%%%%%%%%%%%
% Title page

\maketitle
\makeinnertitle %inside cover page
%%%%%%%%%%%%%%%%%%%%%%%%%%%%%%%%%%%%%%%%%%%%%%%%%%%%%%%%%%%%%%%%%%%%%%
%\quotations
\chapter*{\centerline{QUOTATIONS}}
       \pagenumbering{gobble}
Quotation is optional. If you are mentioning the quotation, remove the title and write your quotation and leave the next page blank so that DEDICATION appears on the right side.

\dedication
Dedication is optional. If you are mentioning the dedication, remove the title and write your dedication text and leave the next page blank so that THESIS CERTIFICATE appears on the right side.
%%%%%%%%%%%%%%%%%%%%%%%%%%%%%%%%%%%%%%%%%%%%%%%%%%%%%%%%%%%%%%%%%%%%%%
% Certificate
\certificate

\vspace*{-0.6in}

\noindent This is to undertake that the thesis titled, \textit{<thesis title>} submitted by me to the Indian Institute of Technology Madras, for the  award  of <Ph.D./M.S.> is  a  bonafide  record  of  the  research  work  done  by  me  under  the supervision of <Name of Guide(s)>. The contents of this thesis, in full or in parts, have not been submitted to any other Institute or University for the award of any degree or diploma.

\begin{singlespace}
  *The research work has been carried out at IIT Madras and <name of organization/institution in which the candidate is an employee>
\end{singlespace}

\begin{singlespace}
  In order to effectively convey the idea presented in this thesis, the following work of other authors was reprinted in the thesis with their permission... \textit{\textbf{[choose one approach and delete this comment]:(if you have extensive referencing here or if the page length is exceeded please share the same in an appendix)'\\
  In case of appendix, please word the certificate as:}}\\
  In order to effectively convey the idea presented in this thesis, the work of other authors was reprinted in the thesis with their permission, as described in Appendix - X, images from internet were taken, as discussed in Appendix -B'
\end{singlespace}

\vspace{-0.5cm}
\begin{itemize}
    \item Figures, Tables from previously published journal papers, to be referenced in the following way:
    \begin{itemize}
        \item Figure/Table XX, page yy: Reprinted from... <Ref details>...with the permission of the authors and/or Publisher name
    \end{itemize}
    \item Images taken from the internet, to be referenced in the following way:
    \begin{itemize}
        \item Photograph by < (name of photographer), distributed under a CC-BY 40 license.
    \end{itemize}
\end{itemize}

\vspace*{-0.3in}

%\hspace{-0.3cm}\textbf{Chennai 600 036} \hfill \textbf{Research Scholar}\\
\hspace{-0.6cm}\textbf{Chennai 600 036} \hspace{7.5cm} \textbf{Research Scholar}\\
\vspace{-1.8cm}\hspace{-0.6cm}\textbf{Date:}\\
\begin{flushright}
\textbf{Research Guide/ Co-ordinator*}
\end{flushright}
\textit{*Applicable to External Registration candidates only}

%\hspace{-1cm}\textcircled{c} 2021 Indian Institute of Technology Madras
\iffalse
\begin{singlespacing}
\hspace*{-0.25in}
\parbox{2.5in}{
\noindent {\bf Prof.~1} \\
\noindent Research Guide \\ 
\noindent Professor \\
\noindent Dept. of Physics\\
\noindent IIT-Madras, 600 036 \\
} 
\hspace*{1.0in} 
%\parbox{2.5in}{
%\noindent {\bf Prof.~S.~C.~Rajan} \\
%\noindent Research Guide \\ 
%\noindent Assistant Professor \\
%\noindent Dept.  of  Aerospace Engineering\\
%\noindent IIT-Madras, 600 036 \\
%}  
\end{singlespacing}
\vspace*{0.25in}
\noindent Place: Chennai\\
Date: 19th January 2009 
\fi
\pagebreak
\afterpage{\blankpage}
%%%%%%%%%%%%%%%%%%%%%%%%%%%%%%%%%%%%%%%%%%%%%%%%%%%%%%%%%%%%%%%%%%%%%%
% List of Publications
\listofpublications

The publications arising out of the work mentioned in this thesis are given as follows
\pagebreak
\afterpage{\blankpage}
%%%%%%%%%%%%%%%%%%%%%%%%%%%%%%%%%%%%%%%%%%%%%%%%%%%%%%%%%%%%%%%%%%%%%%
% Acknowledgements
\acknowledgements

Write the acknowledgement text here.

This version of the MS and Ph.D. thesis template was prepared by Shashank (MS Scholar, CS Department) in 2021.
IIT Madras places on record its gratitude to him and former students who have over the years helped create the template and made it accessible to the students.

%%%%%%%%%%%%%%%%%%%%%%%%%%%%%%%%%%%%%%%%%%%%%%%%%%%%%%%%%%%%%%%%%%%%%%
% Abstract

\abstract

\noindent KEYWORDS: \hspace*{0.5em} \parbox[t]{4.4in}{\LaTeX ; Thesis;
  Style files; Format.}

\vspace*{24pt}

\noindent A \LaTeX\ class along with a simple template thesis are
provided here.  These can be used to easily write a thesis suitable
for submission at IIT-Madras.  The class provides options to format
PhD, MS, M.Tech.\ and B.Tech.\ thesis.  It also allows one to write a
synopsis using the same class file.  Also provided is a BIB\TeX\ style
file that formats all bibliography entries as per the IITM format.

The formatting is as (as far as the author is aware) per the current
institute guidelines.

\pagebreak

%%%%%%%%%%%%%%%%%%%%%%%%%%%%%%%%%%%%%%%%%%%%%%%%%%%%%%%%%%%%%%%%%
% Table of contents etc.

\begin{singlespace}
\tableofcontents
\thispagestyle{empty}

\listoftables
\addcontentsline{toc}{chapter}{LIST OF TABLES}
\listoffigures
\addcontentsline{toc}{chapter}{LIST OF FIGURES}
\end{singlespace}


%%%%%%%%%%%%%%%%%%%%%%%%%%%%%%%%%%%%%%%%%%%%%%%%%%%%%%%%%%%%%%%%%%%%%%

\glossary
The following are some of the commonly used terms in the thesis:\\

\vspace{-0.5cm}

\begin{singlespace}
\begin{tabular}{p{0.2\textwidth} p{0.8\textwidth}}
 
    {\bf Distributed applications} & An application which runs on more than one machine and communicate over the network \\
    \vspace{1cm} & \vspace{1cm}\\
    {\bf Distributed applications} & An application which runs on more than one machine and communicate over the network\\
    
\end{tabular}
\end{singlespace}
%%%%%%%%%%%%%%%%%%%%%%%%%%%%%%%%%%%%%%%%%%%%%%%%%%%%%%%%%%%%%%%%%%%%%%
% Abbreviations
\abbreviations
%\noindent 
\begin{tabbing}
xxxxxxxxxxx \= xxxxxxxxxxxxxxxxxxxxxxxxxxxxxxxxxxxxxxxxxxxxxxxx \kill
\textbf{IITM}   \> Indian Institute of Technology, Madras \\
\textbf{RTFM} \> Read the Fine Manual \\
\end{tabbing}

\pagebreak

%%%%%%%%%%%%%%%%%%%%%%%%%%%%%%%%%%%%%%%%%%%%%%%%%%%%%%%%%%%%%%%%%%%%%%
% Notation

\chapter*{\centerline{NOTATION}}
\addcontentsline{toc}{chapter}{NOTATION}

\vspace{-0.5cm}
\begin{singlespace}
\begin{tabbing}
xxxxxxxxxxx \= xxxxxxxxxxxxxxxxxxxxxxxxxxxxxxxxxxxxxxxxxxxxxxxx \kill
\textbf{$r$}  \> Radius, $m$ \\
\textbf{$\alpha$}  \> Angle of thesis in degrees \\
\textbf{$\beta$}   \> Flight path in degrees \\
\end{tabbing}
\end{singlespace}

\pagebreak
\clearpage

% The main text will follow from this point so set the page numbering
% to arabic from here on.
\pagenumbering{arabic}


%%%%%%%%%%%%%%%%%%%%%%%%%%%%%%%%%%%%%%%%%%%%%%%%%%

% Introduction.

\chapter{INTRODUCTION}
\label{chap:intro}

This document provides a simple template of how the provided
\verb+iitmdiss.cls+ \LaTeX\ class is to be used.  Also provided are
several useful tips to do various things that might be of use when you
write your thesis.

Before reading any further please note that you are strongly advised
against changing any of the formatting options used in the class
provided in this directory, unless you are absolutely sure that it
does not violate the IITM\footnote{its a college name} formatting guidelines.  \emph{Please do not
  change the margins or the spacing.}  If you do change the formatting
you are on your own (don't blame me if you need to reprint your entire
thesis).  In the case that you do change the formatting despite these
warnings, the least I ask is that you do not redistribute your style
files to your friends (or enemies).

It is also a good idea to take a quick look at the formatting
guidelines.  Your office or advisor should have a copy.  If they
don't, pester them, they really should have the formatting guidelines
readily available somewhere.

To compile your sources run the following from the command line:
\begin{verbatim}
% latex thesis.tex
% bibtex thesis
% latex thesis.tex
% latex thesis.tex
\end{verbatim}
Modify this suitably for your sources.

To generate PDF's with the links from the \verb+hyperref+ package use
the following command:
\begin{verbatim}
% dvipdfm -o thesis.pdf thesis.dvi
\end{verbatim}

\section{Package Options}

Use this thesis as a basic\footnote{this is footnote this is footnote this is footnote this is footnote this is footnote this is footnote this is footnote this is footnote} template to format\footnote{this is another footnote} your thesis.  The
\verb+iitmdiss+ class can be used by simply using something like this:
\begin{verbatim}
\documentclass[PhD]{iitmdiss}  
\end{verbatim}

\subsection{Subsection}
To change the title page for different degrees just change the option
from \verb+PhD+ to one of \verb+MS+, \verb+MTech+ or \verb+BTech+.
The dual degree pages are not supported yet but should be quite easy
to add.  The title page formatting really depends on how large or
small your thesis title is.  Consequently it might require some hand
tuning.  Edit your version of \verb+iitmdiss.cls+ suitably to do this.
I recommend that this be done once your title is final.

To write a synopsis simply use the \verb+synopsis.tex+ file as a
simple template.  The synopsis option turns this on and can be used as
shown below.
\begin{verbatim}
\documentclass[PhD,synopsis]{iitmdiss}                                
\end{verbatim}

Once again the title page may require some small amount of fine
tuning.  This is again easily done by editing the class file.

This sample file uses the \verb+hyperref+ package that makes all
labels and references clickable in both the generated DVI and PDF
files.  These are very useful when reading the document online and do
not affect the output when the files are printed.


\section{Example Figures and tables}

Fig.~\ref{fig:iitm} shows a simple figure for illustration along with
a long caption.  The formatting of the caption text is automatically
single spaced and indented.  Table~\ref{tab:sample} shows a sample
table with the caption placed correctly.  The caption for this should
always be placed before the table as shown in the example.

\begin{figure}[htpb]
  \begin{center}
    \resizebox{50mm}{!} {\includegraphics *{iitm}}
    \resizebox{50mm}{!} {\includegraphics *{iitm}}
    \caption {Two IITM logos in a row.  This is also an
      illustration of a very long figure caption that wraps around two
      two lines.  Notice that the caption is single-spaced.}
  \label{fig:iitm}
  \end{center}
\end{figure}

Text after the figure
\newpage
Text before the table

\begin{table}[htbp]
\centering
\caption{My caption}\vspace{10mm}
\label{my-label}
\setlength{\fboxrule}{1pt} %thickness of both borders
\setlength{\fboxsep}{0.4mm}%distance between both borders
\fbox{%
  \setlength{\fboxsep}{0pt}%distance of border from actual table
  \fbox{%
    \begin{tabular}{l|l}
    $x$ & $x^2$ \\ \hline
    1  &  1   \\
    2  &  4  \\
    3  &  9  \\
    4  &  16  \\
    5  &  25  \\
    6  &  36  \\
    7  &  49  \\
    8  &  64  \\ \hline
    \end{tabular}%
  }%
}
\label{tab:sample}
\end{table}
Text after table

\section{Bibliography with BIB\TeX}

I strongly recommend that you use BIB\TeX\ to automatically generate
your bibliography.  It makes managing your references much easier.  It
is an excellent way to organize your references and reuse them.  You
can use one set of entries for your references and cite them in your
thesis, papers and reports.  If you haven't used it anytime before
please invest some time learning how to use it.  

I've included a simple example BIB\TeX\ file along in this directory
called \verb+refs.bib+.  The \verb+iitmdiss.cls+ class package which
is used in this thesis and for the synopsis uses the \verb+natbib+
package to format the references along with a customized bibliography
style provided as the \verb+iitm.bst+ file in the directory containing
\verb+thesis.tex+.  Documentation for the \verb+natbib+ package should
be available in your distribution of \LaTeX.  Basically, to cite the
author along with the author name and year use \verb+\cite{key}+ where
\verb+key+ is the citation key for your bibliography entry.  You can
also use \verb+\citet{key}+ to get the same effect.  To make the
citation without the author name in the main text but inside the
parenthesis use \verb+\citep{key}+.  The following paragraph shows how
citations can be used in text effectively.

More information on BIB\TeX\ is available in the book by
\cite{lamport:86}.  There are many
references~\citep{lamport:86,prabhu:xx} that explain how to use
BIB\TeX.  Read the \verb+natbib+ package documentation for more
details on how to cite things differently.

Here are other references for example.  \citet{viz:mayavi} presents a
Python based visualization system called MayaVi in a conference paper.
\citet{pan:pr:flat-fst} illustrates a journal article with multiple
authors.  Python~\citep{py:python} is a programming language and is
cited here to show how to cite something that is best identified with
a URL.

\section{Other useful \LaTeX\ packages}

The following packages might be useful when writing your thesis.

\begin{itemize}  
\item It is very useful to include line numbers in your document.
  That way, it is very easy for people to suggest corrections to your
  text.  I recommend the use of the \texttt{lineno} package for this
  purpose.  This is not a standard package but can be obtained on the
  internet.  The directory containing this file should contain a
  lineno directory that includes the package along with documentation
  for it.

\item The \texttt{listings} package should be available with your
  distribution of \LaTeX.  This package is very useful when one needs
  to list source code or pseudo-code.

\item For special figure captions the \texttt{ccaption} package may be
  useful.  This is specially useful if one has a figure that spans
  more than two pages and you need to use the same figure number.

\item The notation page can be entered manually or automatically
  generated using the \texttt{nomencl} package.

\end{itemize}

More details on how to use these specific packages are available along
with the documentation of the respective packages.

%%%%%%%%%%%%%%%%%%%%%%%%%%%%%%%%%%%%%%%%%%%%%%%%%%%%%%%%%%%%
% Appendices.

\appendix

\chapter{A SAMPLE APPENDIX}

Just put in text as you would into any chapter with sections and
whatnot.  Thats the end of it.

%%%%%%%%%%%%%%%%%%%%%%%%%%%%%%%%%%%%%%%%%%%%%%%%%%%%%%%%%%%%
% Bibliography.

\begin{singlespace}
  \bibliography{refs}
\end{singlespace}


%%%%%%%%%%%%%%%%%%%%%%%%%%%%%%%%%%%%%%%%%%%%%%%%%%%%%%%%%%%%
% List of papers %did not find list of papers in this format in the new template

\listofpapers

\iffalse
\begin{enumerate}  
\item Authors....  \newblock
 Title...
  \newblock {\em Journal}, Volume,
  Page, (year).
\end{enumerate}  
\fi

\CV

\committee

\end{document}
